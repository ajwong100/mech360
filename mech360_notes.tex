\documentclass{article}
\usepackage{amsmath,amsthm,amsfonts,amssymb,amscd, fancyhdr, color, comment, graphicx, environ, pgfplots}
\usepackage{hyperref} \hypersetup{
    colorlinks=true,
    linkcolor=black,
    filecolor=magenta,      
    urlcolor=blue,
}
\pgfplotsset{compat=1.18}
%Drawing
\usepackage{tikz}

%3D
\usepackage{tikz-3dplot}

%Tikz Library
\usetikzlibrary{angles, quotes, intersections}

%Styles
\tikzset{axis/.style={thick,-latex}}
\tikzset{vec/.style={thick,blue}}
\tikzset{univec/.style={thick,red,-latex}}

%Notation
\usepackage{physics}
\usepackage{bm}
%%PAGE SETUP%%
\usepackage[top=3cm, bottom=4cm, left=3cm, right=3cm]{geometry}
\pagestyle{fancy}
\fancyhead{} % clear all header fields
\fancyhead[LO]{AJ Wong, Eugene Lee}
\fancyhead[RO]{\leftmark} %\leftmark is a chapter title and %\rightmark is a section title
\setlength{\parindent}{0pt} %Makes all indentation values 0 to avoid default indent
%%\parskip=10pt

%%Equation Numbers
\numberwithin{equation}{subsection}

%%COMMANDS%%
\newcommand{\Z}{\mathbb Z}
\newcommand{\C}{\mathbb C}
\newcommand{\R}{\mathbb R}
\newcommand{\Q}{\mathbb Q}
\newcommand{\NN}{\mathbb N}
\newcommand{\PP}{\mathbb P}
\newcommand{\infixiff}{\text{ iff }}
\newcommand{\nobracket}{}
\newcommand{\tmmathbf}[1]{\ensuremath{\boldsymbol{#1}}}
\newcommand{\tmop}[1]{\ensuremath{\operatorname{#1}}}
\newcommand{\tmtextbf}[1]{\text{{\bfseries{#1}}}}
\newcommand{\tmtextit}[1]{\text{{\itshape{#1}}}}
\newcommand{\ans}[1]{\color{blue}\boxed{#1}}
\newcommand{\further}[1]{\textcolor{gray}{#1}}
\newcommand{\lam}{\lambda}

%%MATRICES%%
%%2x2
\newcommand{\cv}[2]{\begin{bmatrix}
  #1\\
  #2\\
\end{bmatrix}}
\newcommand{\m}[4]{\begin{bmatrix}
  #1 & #2\\
  #3 &#4\\
\end{bmatrix}}
\newcommand{\deter}[4]{\begin{vmatrix}
  #1 & #2\\
  #3 &#4\\
\end{vmatrix}}

%%3x3
\newcommand{\mm}[9]{\begin{bmatrix}
  #1 & #2 & #3\\
  #4 & #5 &#6\\
  #7 & #8 & #9\\
\end{bmatrix}}
\newcommand{\deterr}[9]{\begin{vmatrix}
  #1 & #2 & #3\\
  #4 & #5 &#6\\
  #7 & #8 & #9\\
\end{vmatrix}}
\newcommand{\cvv}[3]{\begin{bmatrix}
  #1\\
  #2\\
  #3\\
\end{bmatrix}}
\newcommand{\cvvv}[4]{\begin{bmatrix}
  #1\\
  #2\\
  #3\\
  #4\\
\end{bmatrix}}


%%vector
\newcommand{\vv}[1]{\left\langle #1 \right\rangle}
\newcommand{\bb}[1]{\mathbf{#1}}

%tensor
\newcommand{\tensor}[1]{\underline{\underline{#1}}}

%%partial
\newcommand{\pd}[2]{\frac{\partial #1}{\partial #2}}

%Equation Name
\newcommand{\eqname}[1]{\tag*{#1}}% Tag equation with name


\begin{document}

\begin{titlepage}
    \title{}
    \author{}
    \date{}
    \begin{center}
        \vspace*{3cm}
            
        \Huge
        \textbf{MECH 360 Notes}
            
        \vspace{1cm}
 
        \vspace{1.5cm}
        \Large
        \textbf{By AJ Wong and Eugene Lee} \\                    % <-- author
   
        \vfill
   
        \vspace{1cm}

        \Large
        \today %%Date
    \end{center}
\end{titlepage}

\tableofcontents

\newpage

\section{Transformation of Stress}
Given the 3-D stress tensor
\begin{align}
  \tensor{\sigma} = \mm{\sigma_{xx}}{\sigma_{xy}}{\sigma_{xz}}{\sigma_{yy}}{\sigma_{yx}}{\sigma_{yz}}{\sigma_{zx}}{\sigma_{zy}}{\sigma_{zz}},
\end{align}
the principal stress tensor is given by
\begin{align}
  \tensor{\sigma_p} = \mm{\sigma_1}{0}{0}{0}{\sigma_2}{0}{0}{0}{\sigma_3}.
\end{align}
\textbf{Invarient Properties:}
\begin{itemize}
  \item trace$(\tensor{\sigma})$ = trace$(\tensor{\sigma_p})$
  \item det$(\tensor{\sigma})$ = det$(\tensor{\sigma_p})$
\end{itemize}

\section{Pure Bending}
\subsection{Unsymmetric Bending Analysis}

\textbf{Main Ideas:}
\begin{enumerate}
  \item Moment is a vector - resolving vectors
  \item "Principal" Centroidal axes - Mohr's circle
  \item Superposition - adding stresses
\end{enumerate}
This leads two showing how one unsymmetric bending problem is two symmetric bending problems.
\\

\textbf{Principal Centroidal Axes:}
Where the product of inertia is zero. 
\\

\textbf{What is Product of Inertia?}

Product of Inertia is a measure of body symmetry.

When you talk about symmetry, we talk about planes of symmetry.

If a body is symmetrical about a plane, let's say xy-plane, then $$ I_{xy} = 0$$.


Recall,
\begin{align*}
  I_{xy} = \int xy dA,
\end{align*}
If
\begin{align*}
  I_{zy} = 0, I_y \neq 0, I_z \neq 0,
\end{align*}
then $y$ and $z$ are the principal centroidal axes (PCA). PCA's are mutually orthogonal.
Finding principal planes is analogous to finding principal stresses on the Mohr's circle.
\\

The area moment of inertia is a tensor too. We apply the Mohr's transformations to find PCA.
\\

Any given section possess \emph{principal centroidal axes} even if it is unsymmetric.
Principal centroidal axes can be determined: 
\begin{enumerate}
  \item analytically
  \item or using Mohr's circle.
\end{enumerate}
If $\bb{M}$ is along the principal centroidal axis, the N.A. will be along the axis of $\bb{M}$,
then the equations for symmetric members can be used to compute the stresses.
The principal of superposition is used to determine stresses in general for unsymmetric cases.
\\

Given some a couple momment $\bb{M}$, we have
\begin{align*}
  M_z = M\cos\theta,~~ M_y = M\sin\theta,
\end{align*}
then using superposition,
\begin{align*}
  \sigma_x(y,z) = \frac{-M_z y}{I_z} + \frac{+M_y z}{I_y}.
\end{align*}
Points along the N.A. have no stress, thus let $\sigma_x = 0$, and using
$M_z = M\cos\theta,~~ M_y = M\sin\theta$, we get
\begin{align*}
  y = \underbrace{\left(\frac{I_z}{I_y}\tan\theta\right)}_a z,
\end{align*}
representing a line $y(z)$ with slope $a$.
Letting $\phi$ be the angle between the N.A. and the z-axis gives
\begin{align*}
  \tan\phi = \frac{I_z}{I_y}\tan\theta.
\end{align*}`'
Note, $\phi$ depends on the loading axis. Also,
\begin{align*}
  \tan\phi = \frac{y_n}{z_n},
\end{align*}
where $(y_n, z_n)$ is some point on the NA.
\\

\textbf{Solving Steps:}
If ${\underline{M}}$ is not in the plane of symmetry, we have an unsymmetric bending problem.
\begin{enumerate}
  \item Find PCA - (using Mohr)
  \item Resolve moment vector about PCA
  \item Find/apply $\sigma = \frac{-My}{i}$ for each component
  \item Superposition - (assuming linearly elastic material)
\end{enumerate}

\textbf{Steps solving an example:}
\begin{enumerate}
  \item PCA: found by inspection
  \item Decompose moment into cartesian components
  \item Stress due to asymmetric bending is the sum of the stress due to each moment component
\end{enumerate}

\subsection{Eccentric Loading}
Using superposition
\begin{align*}
  \sigma_x(y,z) = \frac{P}{A} - \frac{M_z y}{I_z} + \frac{M_y z}{I_y}.
\end{align*}


\subsection{Shear Flow}
\begin{itemize}
  \item $q$ is shear flow
  \item Q is first moment of area, units $m^3$ of some partial area of cross-section (dependent on question) about centroid
\end{itemize}
I would like to emphasize that $Q$ is for PART of the cross-sectional area. \\

\begin{itemize}
  \item $I$ is the moment of inertia calculated given by $$\frac{bh^3}{12}$$ and $Ad^2$.
\end{itemize}


For a beam in bending, there is shear flow. \\

I think we've just been ignoring it. \\

\textbf{Steps for shear flow problem}:
\begin{enumerate}
  \item Find moment of Inertia of entire crosssection
  \item Identify the area that you will find Q for. I think the area depends on what the nail of interest is in.
  \item 
\end{enumerate}


\end{document}

